\section{Theoretical Analysis}

\textcolor{red}{ii) the eight equations}\\

In this section, we will perform a theoretical analysis of the \textbf{equations} that determine the behavior of our amp\\

\textcolor{red}{explicar como chagamos ás expressoes}\\

The \textbf{gain} of the amplifier circuit is given by $A_v$ where $G_M$ is the total transcondutance responsible for the gain of the circuit and $g_{out}$ is the conductance (inverse of the resistance) Seen form the output of the circuit.

$$A_v =  \frac{G_M}{g_{out}} \geq 66\hspace{0.1cm}d_b\\[0.125cm]$$

Where:
\begin{equation}
    \begin{cases}
        G_M = g_{m2a} + g_{m1b} \cdot \frac{g_{m7a}}{g_{m7b}} \approx g_{m2a} + 3 \cdot g_{m1b} \\
        g_{out} = g_{op} + g_{on}
    \end{cases}
\end{equation}\\

The total conductance seen from the output can be broken into 2 parts, conductance \textit{up} $g_{op}$ seen from the output node to the upper part of the circuit $V_{dd}$ and conductance \textit{down} $g_{on}$ seen from the output node to the lowest part of the circuit $V_{ss}$.

\begin{equation}
    \begin{cases}
        g_{op} = g_{ds11} \cdot \frac{g_{ds3}}{g_{m3}} \\
        g_{on} = \left( g_{ds2a} + g_{ds7a} \right) \cdot \frac{g_{ds5}}{g_{m5}}
    \end{cases}
\end{equation}\\

The \textbf{gain bandwidth product} can be obtained by the ratio between $G_M$ and $C{out}$ where $C{out}$ is the equivalent capacitance seen from the output node.

$$  GBW = \frac{G_M}{C_{out}} \geqq 100 MHz $$

$C_{out}$ is given by :\\

$$C_{out} = C_L + C_{bd3} + C_{gd3} + C_{bd5} + C_{gd5}$$

\newpage

\textcolor{red}{ver qual é fp2 e fp3 - fpx ou fpz }

in order to guarantee \textbf{stability in a unity-gain closed-loop} configuration we need to define the frequencies
for poles $f_{px}$ and  $f_{pz}$

$$f_{px} = \frac{1}{C_x\cdot r_x} > x  \hspace{2cm}  f_{pz} = \frac{1}{C_z\cdot r_z} > y$$

Thus, we will need to calculate the equivalent capacitance and resistance at node $x$ and $y$ given by:\\

\begin{equation}
    \begin{cases}
        C_x = C_{gd7a} + C_{bd7a}+ C_{gd2a} + C_{bd2a} + C_{gd5} + C_{bs5}\\
        C_z = C_{gs13} + C_{bs13} + C_{bd7b}+ C_{gs7b} \\
        r_z \approx \frac{1}{gm13} \\
        r_x \approx g_{m5} 
    \end{cases}
\end{equation}\\

The \textbf{output swing} of the amplifier can be obtained by the following expression following a margin $V_{margin}$ of 80 mV or 0.08 V :\\

$$OS = V_{DD} - V_{DSsat11} - V_{DSsat3} - V_{DSsat5} - V_{DSsat7a} - 0.08  \geqq  0.5 \hspace{0.2cm} V_{p-p}$$\\

the \textbf{Excess-Noise factor} is the additional noise created by a amplifier with gain $A_v$. A amplifier with excess noise of 1 means that the intrinsic noise is multiplied by 1 therefore, the noise is not made any worse, but any value over 1 means the noise is increased due to the gain, we can calculate the value for Excess-Noise factor $\Gamma$ given by the following equation:\\

$$\Gamma = 1 + \frac{3}{4}\cdot \frac{gm_{8a}}{gm_{1a}} + \frac{1}{4}\cdot \frac{gm_{10}}{gm_{1a}}$$

\newpage

\textbf{Main Goals:}\\

Its intended that the \textbf{power dissipation} $P_D$ is minimized in order to have the best performing low power amp.\\

Where: 

$$P_D = V_{DD} \times \left(2 \times I_B + \dfrac{I_B\cdot 4}{10}\right) $$\\

 We will also need to achieve the best possible \textbf{figure-of-merit FoM} defined by: \\

$$FoM = 1000 \times \dfrac{GBW \times C_L}{P_D}$$\\

\textbf{Adicional constraints:}\\
\begin{itemize}
 \item All channel lengths within $180 nm \leqslant L \leqslant 1.0 \mu m$, in order to avoid pronounced shortchannel effects
 \item All devices should be biased in moderate inversion, for example, with overdrive voltages within 5$0
mV \leqslant V_{DSsat} \leqslant 150 m$V.
\end{itemize}
